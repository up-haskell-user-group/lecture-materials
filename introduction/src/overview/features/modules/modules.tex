\subsection{Packaging in Haskell}

\begin{frame}[fragile]
  \frametitle{Modules}
  \begin{itemize}
    \item A source unit (a file)
    \item Exports selected functions and types.
\begin{lstlisting}
module Some.Module (func1, type1, ...)

func1 :: ...
 
\end{lstlisting}
	 \item Import
	   \begin{itemize}
	     \item Unqualified import:
\begin{lstlisting}
import Some.Module

or

import Some.Module (func1) 
\end{lstlisting}
         \item Qualified import: 

\begin{lstlisting}
import qualified Some.Module

Some.Module.func1 as SM

SM.func1 ...
\end{lstlisting}

      \end{itemize}
  \end{itemize}
  
\end{frame}

\begin{frame}[fragile]
  \frametitle{Packages}
  \begin{itemize}
    \item A package is a unit of distribution.
      \begin{itemize}
        \item Collection of modules (files).
      \end{itemize}
    \item Built via Cabal.
    \item Package meta-data specification:
      \begin{itemize}
        \item globally unique package name,
        \item version id,
        \item dependency list (referrring to package dependencies),
        \item list of exposed modules.
      \end{itemize}
    \item Unit of distribution with Cabal-based metadata and build support.
    \item Collection of modules.  
  \end{itemize}
{\small  
\begin{lstlisting}
import Some.Module

or

import Some.Module (func1) 
\end{lstlisting}
}  
\end{frame}